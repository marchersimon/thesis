\setchapterstyle{kao}
\setchapterpreamble[u]{\margintoc}

\chapter{MIDI}
\labch{mi-theory}

\section{What is MIDI}

\gls{midi} is a protocol for exchanging musical events between devices like synthesizers, keyboards or other electrical instruments. It first appeared as \gls{midi} 1.0 in 1982 and is slowly being replaced by \gls{midi} 2.0, which was introduced in 2020.

Instead of describing the sound of a audio, like e.g. MP3 does, MIDI describes the pitch, volume and duration of every tone. This enables high flexibility for music production, while still being easy to use. For this project, the Interrupter will act as a very simple synthesizer which controls the tesla coil to create the described tones.

As opposed to MIDI streams, which connect two or more devices together and lets them connect in real time, the Interrupter will use \glspl{smf}, which are binary files holding \gls{midi} data.

\section{Standard MIDI File}

Most of the information in the following section was taken from \href{http://www.music.mcgill.ca/~ich/classes/mumt306/StandardMIDIfileformat.html}{www.music.mcgill.ca/~ich/classes/mumt306/StandardMIDIfileformat.html}.

\glspl{smf} are build up of a header chunk and one or more track chunks. The header chunk stores the file format, number of following track chunks and the divisions\sidenote{The time resolution of the file}. The track chunks then hold the actual \gls{midi} information. How these track chunks have to be read depends on the file format:

\begin{tabular}{cll}
    \midrule
    0 & Single Track File Format   & \\
    1 & Multiple Track File Format & All tracks are played simultaneously\\
    2 & Multiple Song File Format  & All tracks are played after one another\\
    \midrule
\end{tabular}

Every track is a series of delta times with a following \gls{midi}, meta, or sysex event.

\subsection{Delta Time}

Delta times specify the time offset to the previous event. They are encoded as Variable Length Quantities, which means they can take up one to four bytes, depending on their size. The MSB % acronym
of each byte shows if another byte follows and the remaining 7 bits are the delta time. An example implementation for decoding a Variable Length Quantity in C++ can be seen in listing

\begin{lstlisting}
uint32_t deltaTime = 0;
for(uint8_t i = 0; i < 4; i++) {
    uint8_t byte = nextByte();
    deltaTime = (deltaTime << 7) | (byte & 0x7F);
    if(~byte & 0x80) {
         break;
    }
}
\end{lstlisting}

The delta time is encoded as number of ticks, which by itself is not very useful. To convert the delta time into seconds, both the division and the tempo are needed. The division is part of the header chunk and the tempo can be set with a meta event. Equation \ref{eq:delta-time} shows how those values can be used to calculate the actual time.

\begin{equation}\label{eq:delta-time}
    \textnormal{time [\textmu s]} = \frac{{\textnormal{tempo} \left[ \frac{\textnormal{\textmu s}}{\textnormal{quarter note}} \right] \cdot \textnormal{delta time } \left[\frac{\textnormal{ticks}}{1}\right]}}{\textnormal{division}  \left[ \frac{\textnormal{ticks}}{\textnormal{quarter note}} \right]}
\end{equation}

\subsection{META-Event}

META-Events are not a necessity to play the music. They show when a track ends or set the tempo. Every META-Event begins with 0xFF, then comes the META-Event type. After the type comes the length of the Meta-Event data. At last comes the actual data. The table below shows most META-Events with the hex index.

\begin{tabular}{|l|l|l|l|}
    \hline
        Type & Event & Type & Event \\ \hline
        0x00 & Sequence number & 0x20 & MIDI channel prefix assignment \\ \hline
        0x01 & Text event & 0x2F & End of track \\ \hline
        0x02 & Copyright notice & 0x51 & Tempo setting \\ \hline
        0x03 & Sequence or track name & 0x54 & SMPTE offset \\ \hline
        0x04 & Instrument name & 0x58 & Time signature \\ \hline
        0x05 & Lyric text & 0x59 & Key signature \\ \hline
        0x06 & Marker text & 0x7F & Sequencer specific event \\ \hline
        0x07 & Cue point & ~ & ~ \\ \hline
    \end{tabular}


\subsection{MIDI-Event}

MIDI-Events carry the actual music data like note on or note off. Every MIDI-Event has an MSB of one. The 4 MSB show the type of the MIDI-Event and the 4 LSB show the MIDI-Channel number. The next one or more bytes show the actual data of the MIDI-Event.

\section{Standard MIDI File}

MIDI files are Binary files. A MIDI file consists of chunks. A chunk begins with 4 ASCII characters after that follow 32 Bit which give the number of bytes in a chunk. MIDI files begin with a header chunk. After the header chunk can be one track chunk or more. The 4 ASCII characters for a header chunk are \textbf{MThd} and for track chunks it is \textbf{MTrk}.

\subsection{Live MIDI ver MIDI Files}

Live MIDI is like a USB connection just another protocol and another cable. MIDI files have all the events in one file and you can play it anywhere if you have the right program.

\begin{center}
\begin{tabular}{@{}cc@{}}
\toprule
\multicolumn{2}{c}{Differences}             \\\midrule
Live MIDI            & MIDI File            \\
cable                & file                 \\
no delta Time        & delta Time           \\\bottomrule
\end{tabular}
\end{center}
\phantom{Hello}

\subsection{Header Chunk}

A header chunk can looks like this.

\begin{tabular}{llll|llll|ll|ll|ll}
\hline
4D & 54 & 68 & 64 & 00     & 00     & 00     & 06     & 00           & 00          & 00                                        & 01                                        & 01            & E0           \\ \hline
M  & T  & h  & d  & \multicolumn{4}{l}{Header Length} & \multicolumn{2}{l}{format} & \multicolumn{2}{l}{\begin{tabular}[c]{|@{}l@{}|}number of \\ track chunks\end{tabular}} & \multicolumn{2}{l}{division}
\end{tabular}

MThd are the four ASCII characters that define, that this is a header chunk. The next four bytes are the header length. The length is always six bytes long. Next two bytes are format, which can be 0 for single track format, 1 for multiple track format and 2 for multiple song format. Currently we only can play \textbf{single track format}. The 2 bytes after the format are the number of track chunks that come after the header chunk because we currently can only play \textbf{single track format} the program can not play more than 1 track chunk. The last two bytes are the division.

\subsection{Track Chunk}

The track chunks begins like the header chunk with 4 ASCII characters which are \textbf{MTrk}. After the 4 characters comes the 4 bytes long length. The length indicates how many bytes are in the track chunk. At last comes the \textbf{track event}.

A track event starts with the \textbf{delta time}. The delta time is the time between one event to the next event. Then comes which type of event it is. There can be three different event types, \textbf{MIDI event}, \textbf{META event} and \textbf{SYSEX event}. We never needed the \textbf{SYSEX event} so there will not be an explanation for it. If the event type is a META event it begins with 0xFF after that comes which META event it is. Then the length of the META event data. At the end comes the META event data. For example if you have a Track event which is META event for instance \textbf{Set Tempo} and a delta time of 0 and the tempo in set tempo is 666667 microseconds per quarter note. It would look like this.

\begin{figure}[h!]
    \centering
    \begin{tabular}{|c|c|c|c|c|}
    0x4D\;\;0x54\;\;0x72\;\;0x6B    & 0x00\;\;0x00\;0x02\;0x22      & 0x00       & 0xFF\;0x51\;0x03             & 0x0A\;0x2C\;0x2B           \\ 
    \hline
    MTrk   & 546 bytes & delta Time & Set Tempo & 666667
    \end{tabular}
\end{figure}