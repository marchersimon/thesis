\setchapterstyle{kao}
\setchapterpreamble[u]{\margintoc}

\chapter{MIDI}
\labch{midi}

\section{What is MIDI}

\glsreset{midi}\gls{midi} is a protocol for exchanging musical events between devices like synthesizers, keyboards or other electrical instruments. It first appeared as \gls{midi} 1.0 in 1982 and is slowly being replaced by \gls{midi} 2.0, which was introduced in 2020.

Instead of describing the sound of a audio, like e.g. MP3 does, MIDI describes the pitch, volume and duration of every tone. This enables high flexibility for music production, while still being easy to use. For this project, the Interrupter will act as a very simple synthesizer which controls the tesla coil to create the described tones.

As opposed to MIDI streams, which connect two or more devices together and lets them connect in real time, the Interrupter will use \glspl{smf}, which are binary files holding \gls{midi} data.

\section{Standard MIDI File}

Most of the information in the following section was taken from \sidecite{midi-spec}.

\glspl{smf} are build up of a header chunk and one or more track chunks. The header chunk stores the file format, number of following track chunks and the divisions\sidenote{The time resolution of the file}. The track chunks then hold the actual \gls{midi} information. How these track chunks have to be read depends on the file format:

\begin{tabular}{cll}
    \midrule
    0 & Single Track File Format   & \\
    1 & Multiple Track File Format & All tracks are played simultaneously\\
    2 & Multiple Song File Format  & All tracks are played after one another\\
    \midrule
\end{tabular}

Every track is a series of delta times with a following \gls{midi}, meta, or sysex event.

\subsection{Events}

Events are instructions for reconstructing the audio of a song. They always consist of an identifier and a variable amount of data. The three types of events are \gls{midi}-Events, Meta-Events, and Sysex-Events, while the latter will not be discussed further, because they are mostly proprietary events from various companies. 

\subsubsection{MIDI-Events}

\gls{midi}-events contain the actual music information, like \cinl{NOTE_ON} or \cinl{NOTE_OFF}. They consist of a 4-bit identifier, a 4-bit channel, and (in most cases) 2 bytes of data. For the example of \cinl{NOTE_ON} and \cinl{NOTE_OFF}, the first byte of the data contains the note number and the second byte contains the velocity, which can be interpreted as its loudness.

\begin{figure}[h!]
  \centering
  \begin{bytefield}[bitwidth=1em]{26}
    \bitbox{1}[bgcolor=gr10]{1} &
    \bitbox{1}[bgcolor=white]{0} &
    \bitbox{1}[bgcolor=white]{0} &
    \bitbox{1}[bgcolor=white]{1} &
    \bitbox{4}{channel} &
    \bitbox[]{1}{} &
    \bitbox{1}[bgcolor=gr10]{0} &
    \bitbox{7}{note number} &
    \bitbox[]{1}{} &
    \bitbox{1}[bgcolor=gr10]{0} &
    \bitbox{7}{velocity}\\
  \end{bytefield}
  \caption{Structure of a NOTE\_ON event}
  \label{fig:note-on}
\end{figure}

The 7-bit note number can range from 0 to 127, which corresponds to the musical notes C-1 to G9. For reference, a common 88-key piano ranges from A0 to C8 (note numbers 21 to 108). To convert this note number into a usable frequency, equation \ref{eq:note-number} can be used.

\begin{equation}\label{eq:note-number}
    f = 440 \cdot 2^{\frac{n - 69}{12}}
\end{equation}

\subsubsection{Meta-Events}

Non-musical information like song names, lyric texts, tempo settings or end of track are contained in meta events. The most common meta events are \cinl{SET_TEMPO} and \cinl{END_OF_TRACK}, which has to be the last event of every track. Every meta event consists of a \cinl{0xFF}, an identifier and a variable amount of data.

\subsection{Delta Time}

Delta times specify the time offset to the previous event. They are encoded as Variable Length Quantities, which means they can take up one to four bytes, depending on their size. The MSB % acronym
of each byte shows if another byte follows and the remaining 7 bits are the delta time. An example implementation for decoding a Variable Length Quantity in C++ can be seen in listing

\begin{lstlisting}
uint32_t deltaTime = 0;
for (uint8_t i = 0; i < 4; i++) {
    uint8_t byte = nextByte();
    deltaTime = (deltaTime << 7) | (byte & 0x7F);
    if(~byte & 0x80) {
         break;
    }
}
\end{lstlisting}

The delta time is the number of ticks, which by itself is not very useful. To convert the delta time into seconds, both the division and the tempo are needed. The division is part of the header chunk and the tempo can be set with a meta event. Equation \ref{eq:delta-time} shows how those values can be used to calculate the actual time.

\begin{equation}\label{eq:delta-time}
    \textnormal{time [\textmu s]} = \frac{{\textnormal{tempo} \left[ \frac{\textnormal{\textmu s}}{\textnormal{quarter note}} \right] \cdot \textnormal{delta time } \left[\frac{\textnormal{ticks}}{1}\right]}}{\textnormal{division}  \left[ \frac{\textnormal{ticks}}{\textnormal{quarter note}} \right]}
\end{equation}

\subsection{Running Status}

One thing all event types have in common is, that their first identifier byte has an \gls{msb} of 1 and their first argument byte has an \gls{msb} of 0. This way, in case the identifier is omitted, the argument is still cleary recognisable as such and cannot be mistaken for the identifier. The running status scheme now allows it to omit the identifier of an event, if it is of the same type as the previous event. For example, after a NOTE\_ON event, consisting of a delta time, an identifier and its argument, a following NOTE\_ON event could only consist of a delta time followed by the argument.

By only by combining this with another trick, which is to turn all NOTE\_OFF events into NOTE\_ON events with a velocity of zero, does this become really useful. Now, for a simple \gls{midi} file, almost the whole file would consist of NOTE\_ON events stringed together and only the first one would need an identifier.

\begin{figure}[h!]
    \centering
    \begin{minipage}{0.45\textwidth}
  \begin{bytefield}[bitwidth=1.5mm]{32}
  \bitbox{8}{\small \(\Delta\)Time}&
  \bitbox{8}{\small N\_ON}&
  \bitbox{8}{\small Note}&
  \bitbox{8}{\small Velocity}\\
  \bitbox{8}{\small \(\Delta\)Time}&
  \bitbox{8}{\small N\_OFF}&
  \bitbox{8}{\small Note}&
  \bitbox{8}{\small Velocity}\\
  \bitbox{8}{\small \(\Delta\)Time}&
  \bitbox{8}{\small N\_ON}&
  \bitbox{8}{\small Note}&
  \bitbox{8}{\small Velocity}\\
  \bitbox{8}{\small \(\Delta\)Time}&
  \bitbox{8}{\small N\_OFF}&
  \bitbox{8}{\small Note}&
  \bitbox{8}{\small Velocity}\\
  \bitbox[]{32}{\(\vdots\)}
\end{bytefield}
\end{minipage}%
    \begin{minipage}{0.1\textwidth}
      \begin{center}
        \raisebox{5mm}{\resizebox{5mm}{!}{\(\Rightarrow\)}}
      \end{center}
\end{minipage}%
    \begin{minipage}{0.45\textwidth}
    \begin{bytefield}[bitwidth=1.5mm]{32}
      \bitbox{8}{\small \(\Delta\)Time}&
      \bitbox{8}{\small N\_ON}&
      \bitbox{8}{\small Note}&
      \bitbox{8}{\small Velocity}\\
      \bitbox{8}{\small \(\Delta\)Time}&
      \bitbox{8}{\small Note}&
      \bitbox{8}{\small 0}&
      \bitbox[]{8}{}\\
      \bitbox{8}{\small \(\Delta\)Time}&
      \bitbox{8}{\small Note}&
      \bitbox{8}{\small Velocity}&
      \bitbox[]{8}{}\\
      \bitbox{8}{\small \(\Delta\)Time}&
      \bitbox{8}{\small Note}&
      \bitbox{8}{\small 0}&
      \bitbox[]{8}{}\\
      \bitbox[]{32}{\(\vdots\)}
      \end{bytefield} 
\end{minipage}
    \caption{Running Status}
    \label{fig:running-status}
\end{figure}

\section{Simplifying the MIDI files}

To not have to be able to decode every single MIDI message, most of which aren't useful to the \gls{midi}-Interrupter anyways, a tool called \emph{tinymid}\sidenote{Its source code can be found at \href{https://github.com/marchersimon/tinymid}{github.com/marchersimon/tinymid}.} was used. It removes all unnecessary events, moves them to channel 0, converts them to running status if possible, and most importantly, it merges all tracks of a multiple track file into a single track to make it a single track file. For this the whole file has to be decoded at once and all events sorted by their absolute delta time, which would have been much harder to do with the limited libraries and debugging capabilities available for AVR C++.