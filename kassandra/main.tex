\setchapterstyle{kao}
\setchapterpreamble[u]{\margintoc}

\setcounter{chapter}{9}

\chapter{Casing}
\labch{db-section-1}

AAAAA AAAAA AAAAA AAAAA AAAAA AAAAA AAAAA AAAAA AAAAA AAAAA AAAAA AAAAA AAAAA AAAAA AAAAA AAAAA AAAAA AAAAA AAAAA AAAAA AAAAA AAAAA AAAAA AAAAA AAAAA AAAAA AAAAA AAAAA AAAAA AAAAA AAAAA AAAAA AAAAA AAAAA AAAAA AAAAA AAAAA AAAAA AAAAA AAAAA AAAAA AAAAA AAAAA AAAAA AAAAA AAAAA AAAAA AAAAA AAAAA AAAAA AAAAA AAAAA AAAAA AAAAA AAAAA AAAAA AAAAA AAAAA AAAAA AAAAA AAAAA AAAAA AAAAA AAAAA AAAAA

\sidenote{Printed Circuit Board}
\todo[backgroundcolor=blue,
bordercolor=green!60!white, textcolor=red]{PDF Coil Construction aka Simons Buch auf Telegram}

\section{Envision}

AAAAA AAAAA AAAAA AAAAA AAAAA AAAAA AAAAA AAAAA AAAAA AAAAA AAAAA AAAAA AAAAA AAAAA AAAAA AAAAA AAAAA AAAAA AAAAA AAAAA AAAAA AAAAA AAAAA AAAAA AAAAA AAAAA AAAAA AAAAA AAAAA AAAAA AAAAA AAAAA AAAAA AAAAA AAAAA AAAAA AAAAA AAAAA AAAAA AAAAA AAAAA AAAAA AAAAA AAAAA AAAAA AAAAA AAAAA AAAAA AAAAA AAAAA AAAAA AAAAA AAAAA AAAAA AAAAA AAAAA AAAAA AAAAA AAAAA AAAAA AAAAA AAAAA AAAAA AAAAA AAAAA
\todo[backgroundcolor=blue,
bordercolor=green!60!white, textcolor=red]{Grob Design vorlage mit bild(in Inventor) aber nicht wie wir es umsetzen(aka die steher). Alles Modular sein tun}

\section{Materials}

AAAAA AAAAA AAAAA AAAAA AAAAA AAAAA AAAAA AAAAA AAAAA AAAAA AAAAA AAAAA AAAAA AAAAA AAAAA AAAAA AAAAA AAAAA AAAAA AAAAA AAAAA AAAAA AAAAA AAAAA AAAAA AAAAA AAAAA AAAAA AAAAA AAAAA AAAAA AAAAA AAAAA AAAAA AAAAA AAAAA AAAAA AAAAA AAAAA AAAAA AAAAA AAAAA AAAAA AAAAA AAAAA AAAAA AAAAA AAAAA AAAAA AAAAA AAAAA AAAAA AAAAA AAAAA AAAAA AAAAA AAAAA AAAAA AAAAA AAAAA AAAAA AAAAA AAAAA AAAAA AAAAA
\todo[backgroundcolor=blue,
bordercolor=green!60!white, textcolor=red]{Welche veraussetzung die materialen haben müssen und warum gewisse nicht in frage kommen. Berechnungen auch}

\section{Structure for the Coils}

AAAAA AAAAA AAAAA AAAAA AAAAA AAAAA AAAAA AAAAA AAAAA AAAAA AAAAA AAAAA AAAAA AAAAA AAAAA AAAAA AAAAA AAAAA AAAAA AAAAA AAAAA AAAAA AAAAA AAAAA AAAAA AAAAA AAAAA AAAAA AAAAA AAAAA AAAAA AAAAA AAAAA AAAAA AAAAA AAAAA 
\todo[backgroundcolor=blue,
bordercolor=green!60!white, textcolor=red]{Das Gerüst für die zwei spulen. Steher Desgin und funktion. Spitze mit Kupfer mit Spindel. Vlt erstes design mit Löt}

\section{It's not a Bug, it's a feature}

AAAAA AAAAA AAAAA AAAAA AAAAA AAAAA AAAAA AAAAA AAAAA AAAAA AAAAA AAAAA AAAAA AAAAA AAAAA AAAAA AAAAA AAAAA AAAAA AAAAA AAAAA AAAAA AAAAA AAAAA AAAAA AAAAA AAAAA AAAAA AAAAA AAAAA AAAAA AAAAA AAAAA AAAAA AAAAA AAAAA 
\todo[backgroundcolor=blue,
bordercolor=green!60!white, textcolor=red]{der Bug mit der Spitze aka Dach hohenverstellbar und wie. Auch Tests mit verschiedenen Anzahl an Kanten. Und aus was für einen Material, isolierung und so weita}

\section{Board Casing}

AAAAA AAAAA AAAAA AAAAA AAAAA AAAAA AAAAA AAAAA AAAAA AAAAA AAAAA AAAAA AAAAA AAAAA AAAAA AAAAA AAAAA AAAAA AAAAA AAAAA AAAAA AAAAA AAAAA AAAAA AAAAA AAAAA AAAAA AAAAA AAAAA AAAAA AAAAA AAAAA AAAAA AAAAA AAAAA AAAAA 
\todo[backgroundcolor=blue,
bordercolor=green!60!white, textcolor=red]{warum hexgon?}

\section{Final}

AAAAA AAAAA AAAAA AAAAA AAAAA AAAAA AAAAA AAAAA AAAAA AAAAA AAAAA AAAAA AAAAA AAAAA AAAAA AAAAA AAAAA AAAAA AAAAA AAAAA AAAAA AAAAA AAAAA AAAAA AAAAA AAAAA AAAAA AAAAA AAAAA AAAAA AAAAA AAAAA AAAAA AAAAA AAAAA AAAAA 
\todo[backgroundcolor=blue,
bordercolor=green!60!white, textcolor=red]{vlt was Ozonen machen wie zB desinfezieren}

\chapter{Printed Circuit Board}

\section{Why no Breadboards and Solutions}

AAAAA AAAAA AAAAA AAAAA AAAAA AAAAA AAAAA AAAAA AAAAA AAAAA AAAAA AAAAA AAAAA AAAAA AAAAA AAAAA AAAAA AAAAA AAAAA AAAAA AAAAA AAAAA AAAAA AAAAA AAAAA AAAAA AAAAA AAAAA AAAAA AAAAA AAAAA AAAAA AAAAA AAAAA AAAAA AAAAA 
\todo[backgroundcolor=blue,
bordercolor=green!60!white, textcolor=red]{der fakt das es bei dem hohen frequenzbereich und den kleinen capazitäten(datenblätter für breadboards). Unsere mit einzelnen Bauteilen bestückte Platinen}

\section{Essential Components}

AAAAA AAAAA AAAAA AAAAA AAAAA AAAAA AAAAA AAAAA AAAAA AAAAA AAAAA AAAAA AAAAA AAAAA AAAAA AAAAA AAAAA AAAAA AAAAA AAAAA AAAAA AAAAA AAAAA AAAAA AAAAA AAAAA AAAAA AAAAA AAAAA AAAAA AAAAA AAAAA AAAAA AAAAA AAAAA AAAAA 
\todo[backgroundcolor=blue,
bordercolor=green!60!white, textcolor=red]{geht in der simulierung aba nicht in da parxis. Design choices bei selbst erstellten bauteile. Massefläche warum und Desgin fehler bei Gehäuse bauteilen wie bei Spule und capazitator}

\section{Structure for the Coils}

AAAAA AAAAA AAAAA AAAAA AAAAA AAAAA AAAAA AAAAA AAAAA AAAAA AAAAA AAAAA AAAAA AAAAA AAAAA AAAAA AAAAA AAAAA AAAAA AAAAA AAAAA AAAAA AAAAA AAAAA AAAAA AAAAA AAAAA AAAAA AAAAA AAAAA AAAAA AAAAA AAAAA AAAAA AAAAA AAAAA 

\chapter{Future Ideas and Optimizations}

