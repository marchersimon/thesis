\chapter{Printed Circuit Board}

\glspl{pcb} playe an essential role in this project. Breadboards and other prototyping techniques tend to cause issues with parasitic inductances and capacitances and do not work with \gls{smd} components, so \glspl{pcb} were the only way to build up the circuits. Once a \gls{pcb} has been produced, it works reliable and usually does not add any erratic effects. On the downside, however, \glspl{pcb} take a lot of time to design, etch, and solder, and once they are manufactured, they are very inflexible. This means that a new \gls{pcb} had to be made for every prototype..

\section{Component Selection}

Because almost none of the needed components were on hand in the school laboratories, they had to be ordered online. Due to the unique requirements of some parts, like the MOSFET or the \gls{vco}, they were rather hard to find and either very costly or came in an impractical package.

Since all the essential components were only available as \glspl{smd}, most other components were also shifted to \gls{smt} for consistency.


\subsection{ICs}

Most ICs were only available in one package and did not leave much flexibility. The MOSFET driver, \gls{vco}, latch, AND gate, and MOSFET for the phototransistor came in standard SOT and SOIC packages which were easy to solder since they have the pins on their side. The class-E MOSFET and the voltage regulator were somewhat troublesome because their packages had a big metal area on the underside, only indented for one-time reflow soldering. Unfortunately, the class-E MOSFET still needed to be replaced multiple times during the testing process.

\subsection{Passive Elements}

All capacitors and resistors have been ordered in a 0805 package because it is not too big but just big enough to be easy to work with. The small size compared to \glspl{thd} does, however, negatively affect the power rating of resistors and the voltage rating of capacitors, so it had to be made sure that they still operated within the safe operating range. For the capacitors, X7R devices had been chosen, which means that their operating temperature ranges from -55\textdegree C to 125\textdegree C with at most 15\% capacitance change over this range\textsuperscript{\sidecite{epci}}.

%\subsection{Connectors}



% Took some time to find all components

% SMD
% This is the final list of selected components:

\begin{tabular}{@{}lll@{}}
    \toprule
    \textbf{Part name} & \textbf{Footprint} & \textbf{Description}\\\midrule
    BSC12DN20N & PG-TDSON-8 & MOSFET\\
    IX4310 & SOT-23 & MOSFET driver\\
    LTC1799 & SOT-23 & Oscillator\\
    LM7805 & TO252 & Voltage regulator\\
    PKE3316 & Custom THT & boost converter\\
    74HC72 & SOIC127 & D-type latch\\
    TC7SZ08F & SOT-32 & AND gate\\
    RUM001L02 & SOT-723 & MOSFET\\
    47\(\mu\)H choke & Custom SMD &\\
    Various Capacitors & 0805 &\\
    Various Resistors & 0805 &\\
    Potentiometers & PT-10&\\
    Fuse 250mA & 1206 &\\
    Clamp & ? &\\
    Connector & CON06 &\\
    \bottomrule
\end{tabular}

\section{Physical arrangement}
\label{sec:physical-arrangement}

Two things about the placement of the electronics were already certain - that the class-E amplifier and its surroundings were placed in the center of the casing and that the connectors and controls had to be mounted on the side of the casing. The question is how to connect them. Using loose cables would quickly become unclear and cause noise to be picked up by the unamplified interrupter signal from the phototransistor. This could be solved by bundling all signals directly onto a \gls{pcb} and leading them to the main \gls{pcb} via a six pin flat ribbon cable. Additionally, this side board could amplify the interrupter signal to 12V to make it more resilient to induction. 


% Connection to the Coil
% Connection between the PCBs
% Connection to the Perpherals

\section{PCB Layout}

%\section{Manufacturing}