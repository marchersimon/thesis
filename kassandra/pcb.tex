\chapter{Printed Circuit Board}

\glspl{pcb} played an essential role in this project. Because breadboards and other prototyping techniques tend to cause issues with parasitic inductance and capacitance and don't work with \gls{smd} components, \glspl{pcb} were the only way of connecting the components. Once a \gls{pcb} has been produced, it works reliable and usually don't add any erratic effects. On the downside, however, \glspl{pcb} take a lot of time to design, etch, and solder and once they're manufactured, they're very inflexible. This means that for every prototybe, a new \gls{pcb} had to be made.

\section{Component Selection}

Because almost none of the needed components where on hand in the school laboratories, they had to be ordered online. Due to the special requirements of some parts, like the MOSFET or the \gls{vco}, they were rather hard to find and either very costly or came in an impractical package.

Since all the key components were only available as \glspl{smd}, most of the other components were also shifted to \gls{smt} for consistency.


\subsection{ICs}

Most of the ICs were only available in one package and didn't leave much flexibility. The MOSFET driver, \gls{vco}, latch, AND gate and MOSFET for the phototransistor came in standard SOT and SOIC packages which were easy to solder, since they have the pins on their side. The class E MOSFET and the voltage regulator however were a bit more troublesome, because their packages had a big metal area on the underside, which was only indented for one-time reflow soldering. Especially the class E MOSFET however, needed to be replaced multiple times during the testing process.

\subsection{Passive Elements}

All capacitors and resistors have been ordered in a 0805 package, because it's not too big, but just big enough to be easy to work with. The small size compared to \glspl{thd} does however negatively affect the power rating of resistors and the voltage rating of capacitors, so it had to be made sure, that they still operated within the safe operating range. For the capacitors, X7R devices had been chosen, which means that their operating temperature ranges from -55\textdegree C to 125\textdegree C with at most 15\% capacitance change over this range\textsuperscript{\sidecite{epci}}.

%\subsection{Connectors}



% Took some time to find all components

% SMD
% This is the final list of selected components:

\begin{tabular}{@{}lll@{}}
    \toprule
    \textbf{Part name} & \textbf{Footprint} & \textbf{Description}\\\midrule
    BSC12DN20N & PG-TDSON-8 & MOSFET\\
    IX4310 & SOT-23 & MOSFET driver\\
    LTC1799 & SOT-23 & Oscillator\\
    LM7805 & TO252 & Voltage regulator\\
    PKE3316 & Custom THT & boost converter\\
    74HC72 & SOIC127 & D-type latch\\
    TC7SZ08F & SOT-32 & AND gate\\
    RUM001L02 & SOT-723 & MOSFET\\
    47\(\mu\)H choke & Custom SMD &\\
    Various Capacitors & 0805 &\\
    Various Resistors & 0805 &\\
    Potentiometers & PT-10&\\
    Fuse 250mA & 1206 &\\
    Clamp & ? &\\
    Connector & CON06 &\\
    \bottomrule
\end{tabular}

\section{Physical arrangement}
\label{sec:physical-arrangement}


\section{PCB Layout}

%\section{Manufacturing}