\chapter{Printed Circuit Board}

\glspl{pcb} play an essential role in this project. Breadboards and other prototyping techniques tend to cause issues with parasitic inductances and capacitances and do not work with \gls{smd} components, so \glspl{pcb} were the only way to build up the circuits. Once a \gls{pcb} has been produced, it works reliable and usually does not add any erratic effects. On the downside, however, \glspl{pcb} take a lot of time to design, etch, and solder, and once they are manufactured, they are very inflexible. This means that a new \gls{pcb} had to be made for every prototype..

\section{Component Selection}

Because almost none of the needed components were on hand in the school laboratories, they had to be ordered online. Due to the unique requirements of some parts, like the MOSFET or the \gls{vco}, they were rather hard to find and either very costly or came in an impractical package.

Since all the essential components were only available as \glspl{smd}, most other components were also shifted to \gls{smt} for consistency.


\subsection{ICs}

Most ICs were only available in one package and did not leave much flexibility. The MOSFET driver, \gls{vco}, latch, AND gate, and MOSFET for the phototransistor came in standard SOT and SOIC packages which were easy to solder since they have the pins on their side. The class-E MOSFET and the voltage regulator were somewhat troublesome because their packages had a big metal area on the underside, only indented for one-time reflow soldering. Unfortunately, the class-E MOSFET still needed to be replaced multiple times during the testing process. % flipflop & RUM

\subsection{Passive Elements}

All capacitors and resistors have been ordered in a 0805 package because it is not too big but just big enough to be easy to work with. The small size compared to \glspl{thd} does, however, negatively affect the power rating of resistors and the voltage rating of capacitors, so it had to be made sure that they still operated within the safe operating range. For the capacitors, X7R devices had been chosen, which means that their operating temperature ranges from -55\textdegree C to 125\textdegree C with at most 15\% capacitance change over this range\textsuperscript{\sidecite{epci}}. % potentiometer

\subsection{Peripherals}

To provide the necessary power and lead the MIDI-Interrupter signal to the \gls{pcb} as well as the ability to be turned on and off without being plugged out, one side of the \gls{pcb} enclosure has been dedicated as a control panel. % Possibly wordy sentence
For the power, a 5.5 x 2.1 mm barrel plug was used because of its widespread usage. To connect the fiber optic cable, the phototransistor has been mounted directly into the casing, so it is still accessible from the outside. Additionally, a keyswitch\sidenote{This way, only authorized people are able to turn on the Tesla coil and feel important} is used to turn on the power, and an ON-OFF-ON switch either routes the output of the phototransistor for interrupted operation, a constant 12V signal for continuous operating, or a 0V signal for no operation to the interrupt input.

\subsection{Miscellaneous Parts}

Since the four peripheral components cannot be mounted directly on the \gls{pcb}, because of its placement they had to be connected via wires. To cleanly organize those connections and be able to efficiently disconnect them all at once, a 6-pin connector was used. The single strands of the flat ribbon cable were then separated to connect to the switches and plugs.

A fuse was added on the Vcc rail to protect the circuit and the power supply from overcurrent. To save space and adhere to the \gls{smd}-only strategy, a 1206 package was used.

The boost converter for the 12\,V to 50\,V conversion was a tricky and very costly part. There were only very few parts for the desired voltage range and most of them were designed for hundreds of watts and came with a corresponding price tag. The final decision fell on the \emph{PKE3316HPI} from \emph{Flex Power Modules} which costs \emph{only} 33\,€.

% connectors to coil

\begin{tabular}{@{}lll@{}}
    \toprule
    \textbf{Part name} & \textbf{Footprint} & \textbf{Description}\\\midrule
    BSC12DN20N & PG-TDSON-8 & MOSFET\\
    IX4310 & SOT-23 & MOSFET driver\\
    LTC1799 & SOT-23 & Oscillator\\
    LM7805 & TO252 & Voltage regulator\\
    PKE3316 & Custom THT & boost converter\\
    74HC72 & SOIC127 & D-type latch\\
    TC7SZ08F & SOT-32 & AND gate\\
    RUM001L02 & SOT-723 & MOSFET\\
    47\(\mu\)H choke & Custom SMD &\\
    Various Capacitors & 0805 &\\
    Various Resistors & 0805 &\\
    Potentiometers & PT-10&\\
    Fuse 250mA & 1206 &\\
    Clamp & ? &\\
    \bottomrule
\end{tabular}

\section{Wiring}

\subsection{Peripherals}

\subsection{Coil connection}

\section{PCB Layout}

\begin{figure}[h!]
    \centering
    \includegraphics[width=\textwidth]{kassandra/resources/Tesla6Final.PNG}
    \caption{Final PCB Layout}
    \label{fig:tesla_layout}
\end{figure}

\subsection{Ground Planes}

Ground planes help reducing EMI and

The PCB can be separated into two logic parts - the analog class-E part and the digital part. While the analog part has a relatively high \gls{emc}, the digital part needs to be protected. 

\subsection{High Frequency Stuff}

\section{Manufacturing}

While \glspl{pcb} are typically ordered from an external company, this was not possible in this case. Having them produced is very costly and takes more time, than manufacturing them in-house. The workshop's \gls{pcb} department, although basic, has been a great alternative, especially because of the many prototypes that had to be produced. Also, the flexibility of being able to control every single manufacturing step is impossible to reach when ordering \glspl{pcb}. On the downside, it lacked some professional equipment, like a solder resist station or a stencil cutter. This made the component assembly process significantly harder and more time consuming.

The used manufacturing technique was acidic etching % check if really acidic and not alkaline (https://www.orientdisplay.com/knowledge-base/pcb-basics/what-are-pcb-etching-techniques/)
and the available \gls{pcb} blanks were single or double layered FR-1 blanks. % probably not FR-1

\subsection{Exposure}

The first step was to chemically apply a footprint of the traces to the PCB to the photoresist layer of the PCB. This was done with a designated UV exposure unit. The part of the positive photoresist which gets hit by UV light becomes soluble, while the part covered by a photomask stays insoluble. This photomask was just a semi-transparent sheet of paper with traces printed onto it. It had to be made sure to use as much ink as possible to make it as lightproof as possible.

The photomask for the top and bottom layer had to be carefully aligned, and it had to be made sure that the printed side of the paper faced the PCB. To further eliminate any possible gap between the ink and the PCB, the exposure unit comes with a vacuum pump to press the paper onto the PCB. Before removing the protective film from the PCB, the room has to be dimmed to prevent light from hitting the photoresist beforehand. The exposure process itself then takes about 90 seconds.

\subsection{Stripping, Etching}

Stripping is the process which removes the UV exposed part of the photoresist. The PCB gets mounted in a fixture and sprayed with a \texttt{insert correct name here} at room temperature for one minute. During this process the room has to be kept dim and can only be lit up afterwards. To clean the \gls{pcb} from any remaining \texttt{insert correct name here}-traces, it has to be cleaned with water. % last sentence is weird

The next step is etching away the naked copper areas. This is done by immersing the PCB into a stream of foamed \texttt{insert correct name here}, which was heated up to 50\textdegree C. It takes about 10 minutes for the \texttt{insert correct name here} to fully dissolve the copper, but that slightly changes from time to time and it has to be made sure that no unwanted copper is left manually. Finally, the remaining, unexposed photoresist has to be removed with propanone\sidenote{Formally known as acetone.}.

\subsection{Holes and Vias}



\subsection{Assembly}

\subsection{Potential Problems}