\chapter*{Affidavit}
\addcontentsline{toc}{chapter}{Affidavit}

I declare in lieu of an oath that I have written the present thesis independently
and without outside help other than the stated sources, and that I have made the
passages taken from these sources recognisable as such.

\vfill

\parbox{\textwidth}{
    \parbox{4cm}{
      \centering
      \rule{4cm}{1pt}\\
       Marcher Simon
    }
    \hfill
    \parbox{4cm}{
      \centering
      \rule{4cm}{1pt}\\
      Peyer Kassandra
    }
    \hfill
    \parbox{4cm}{
      \centering
      \rule{4cm}{1pt}\\
      Hamrle Felix
    }
}

\chapter*{Abstract}
\addcontentsline{toc}{chapter}{Abstract}

During the last few decades, Tesla coils have seen a major upwind in popularity. Its impact in the scientific community reached far beyond the indented purpose of high voltage and \resizebox{2mm}{!}{X}-ray generation. This diploma thesis guides through the design process and lays out the inner working of those devices. As opposed to most existing Tesla coils operating at 100 \\- 500\,kHz, this coil's operating frequency is located in the lower MHz band, which poses a series of interesting challenges to overcome. Most noteworthy being the power generation, that required the use of a highly-efficient class-E amplifier. Simulation tools were used \hspace{10pt}in order to verify calculated values. A proper casing and a PCB was designed to reach nominal consumer product quality. An external, AVR-based MIDI-Interrupter uses the well- known MIDI protocol to let the Tesla coil play simple music. In its final stage, it was able to produce arcs up to 10\,mm with only a few watts of input power.

\vspace{4cm}

\begingroup
\let\clearpage\relax
\chapter*{Abriss}
\endgroup


In den letzten Jahrzehnten haben Tesla-Spulen einen großen Aufwind in der Popularität erfahren. Seine Wirkung in der wissenschaftlichen Gemeinschaft reichte weit über den beabsichtigten Zweck der Hochspannungs- und Röntgenstrahlenerzeugung hinaus. Diese Diplomarbeit führt durch den Entwurfsprozess und legt die Innenleben dieser Geräte. Im Gegensatz zu den meisten bestehenden Tesla-Spulen, die bei 100 - 500 kHz arbeiten, liegt die Betriebsfrequenz dieser Spule im unteren MHz-Band, was eine Reihe interessanter Herausforderungen mit sich bringt, die es zu bewältigen gilt. Am bemerkenswertesten ist die Stromerzeugung, die den Einsatz eines hocheffizienten Klasse-E-Verstärkers erforderte. Zur Überprüfung der berechneten Werte wurden Simulationswerkzeuge eingesetzt. Ein geeignetes Gehäuse und eine Leiterplatte wurden entwickelt, um die nominelle Verbraucherproduktqualität zu erreichen. Ein externer, AVR-basierter MIDI-Interrupter nutzt die gut bekanntes MIDI-Protokoll, um die Tesla-Spule einfache Musik spielen zu lassen. In seiner Endstufe konnte es mit nur wenigen Watt Eingangsleistung Lichtbögen bis zu 10 mm erzeugen.