\newpage
\mbox{}
\newpage

\begin{center}
\textbf{\huge Affidavit}
\end{center}

\addcontentsline{toc}{chapter}{Affidavit}

\vspace*{2cm}
\begin{center}
\begin{minipage}{0.9\textwidth}
\begin{center}
I declare in lieu of an oath that I have written the present thesis independently and without outside help other than the stated sources, and that I have made the passages taken from these sources recognisable as such.
\end{center}
\end{minipage}
\end{center}


\vfill

\parbox{\textwidth}{
    \parbox{4cm}{
      \centering
      \rule{4cm}{1pt}\\
    Simon Marcher
    }
    \hfill
    \parbox{4cm}{
      \centering
      \rule{4cm}{1pt}\\
      Kassandra Peyer
    }
    \hfill
    \parbox{4cm}{
      \centering
      \rule{4cm}{1pt}\\
      Felix Hamrle
    }
}

\newpage
\mbox{}
\newpage

\chapter*{Abstract}
\addcontentsline{toc}{chapter}{Abstract}

During the last few decades, Tesla coils have seen a major upwind in popularity. Its impact in the scientific community reached far beyond the indented purpose of high voltage and \resizebox{2mm}{!}{X}-ray generation. This diploma thesis guides through the design process and lays out the inner working of those devices. As opposed to most existing Tesla coils operating at 100 \hspace{10pt}- 500\,kHz, this coil's operating frequency is located in the lower MHz band, which poses a series of interesting challenges to overcome. Most noteworthy being the power generation, that required the use of a highly-efficient class-E amplifier. Simulation tools were used \hspace{10pt}in order to verify calculated values. A proper casing and a PCB was designed to reach nominal consumer product quality. An external, AVR-based MIDI-Interrupter uses the well- known MIDI protocol to let the Tesla coil play simple music. In its final stage, it was able to produce arcs up to 10\,mm with only a few watts of input power.

\vspace{3cm}

\begingroup
\let\clearpage\relax
\chapter*{Zusammenfassung}
\endgroup

In den letzten Jahrzehnten hat die Teslaspule an großer Beliebtheit gewonnen. Ihr Einfluss in der wissenschaftlichen Forschung geht weit über ihren ursprünglichen Zweck der Hochspannungs- und Röntgenstrahlenerzeugung hinaus. Diese Diplomarbeit behandelt einen Designprozess und zeigt den Aufbau und die Funktionsweise eines solchen Geräts auf. Im Gegensatz zu den meisten Teslaspulen, welche im Bereich von 100 - 500\,kHz arbeiten, liegt diese Spule im niedrigen MHz Bereich, was zu einer Menge an interessanten Herausforderungen führt. Eine dieser Herausforderungen, ist die Leistungserzeugung, welche nur mit einem hocheffizienten Klasse-E Verstärker bewältigt werden kann. Ergebnisse von Berechnungen wurden durch Simulationen bestätigt. Um die Qualität eines Endverbraucherprodukts zu erreichen, wurde ein geeignetes Gehäuse mit einer Printplatte konstruiert. Um mit der Teslaspule Musik erzeugen zu können, wurde ein externer AVR-basierter MIDI-Interrupter programmiert. Am Ende konnte die Teslaspule Lichtbögen von bis zu 10\,mm mit einer Eingangsleistung von nur wenigen Watt erzeugen.

\newpage
\mbox{}
\newpage